\documentclass{article}[12pt]
\usepackage{booktabs}
\usepackage{adjustbox}
\usepackage{caption}
\usepackage{graphicx} % Required for inserting images
\usepackage{amsmath}

\usepackage{natbib}


\usepackage{natbib}
\usepackage[a4paper,margin=1in]{geometry} % Set margins to 1 inch on all sides
\usepackage{setspace} % For controlling line spacing
\onehalfspacing % Set line spacing to 1.5

\doublespacing

\title{Incarceration and Economic Disadvantage: Unveiling the Impact on Personal Income} 
\author{Group 5}
\date{April 2024}

\begin{document}
\maketitle

\section{Introduction} %264 + some citation (news, review, etc.)
Incarceration is increasingly recognised as a significant contributor to individual poverty, impacting not only through direct financial penalties and the loss of personal income associated with imprisonment but also via the subsequent social exclusion and discrimination that formerly incarcerated individuals often face upon release. This research investigates the critical question: How does incarceration affect the personal income of those sentenced? Answering this question is economically vital as it provides insights into the broader impacts of incarceration policies on income inequality, which has become markedly pronounced in recent decades. Motivated by the urgent need to break the cycle of poverty and crime, this project aims to isolate the specific economic effects of incarceration from other confounding factors. Low income often drives individuals to engage in criminal activities as a survival strategy when traditional employment opportunities are scarce, necessitating robust methodologies to assess incarceration's true impact. To address this challenge, our study employs propensity score matching, a method that simulates a randomised control trial by matching sentenced individuals with non-incarcerated counterparts who share similar observable characteristics. This approach effectively mitigates the selection bias inherent in the relationship between income and criminal activity, allowing for a clearer assessment of how incarceration impacts individual income. The preliminary findings reveal that incarceration significantly reduces the income potential of the incarcerated individuals. This result underscores the economic importance of revisiting and potentially reforming incarceration policies to support not only the reintegration but also the economic stability of formerly incarcerated individuals.
\par


\section{Literature Review} % more literature 387 
The socioeconomic impact of incarceration, particularly on post-incarceration income, has been extensively studied within labor economics. \cite{western2000effects} find that post-incarceration, the wages of former inmates grow at a 30\% slower rate compared to non-inmates. While this body of research highlights the negative consequences of incarceration on income, the causal nature of this relationship remains contested due to potential confounding variables \citep{apel2010impact}. Longitudinal datasets, such as the National Longitudinal Survey of Youth (NLSY79), have been pivotal in tracking individuals’ labor market participation over time. Studies leveraging such data \citep{western2006punishment} have illustrated the long-term wage penalties associated with incarceration. However, these analyses often face challenges in isolating the causal impact of incarceration from the pre-existing differences between those who are and are not incarcerated. To overcome these methodological challenges, Propensity Score Matching (PSM) has been utilised as a technique to simulate random assignment in observational studies. The method, first introduced by \cite{rosenbaum1983central}, has gained acceptance for its ability to control for confounding variables that may affect both the likelihood of incarceration and subsequent income outcomes \citep{dehejia2002propensity}. However, the validity of PSM is often challenged due to the risk of omitted variable bias \citep{smith2005does}, underscoring the methodological difficulty in isolating the direct effects of incarceration. Recent advancements have seen PSM applied to the NLSY79 data to explore various labor market outcomes \citep{hotchkiss2018effects}. The richness of the NLSY79 data provides an array of observable characteristics, making it an ideal candidate for PSM. However, there has been a lack of research specifically utilising PSM to investigate the causal relationship between incarceration and subsequent income levels. This study aims to address this gap by employing PSM with the NLSY79 data, focusing on matching incarcerated individuals with non-incarcerated counterparts based on a carefully selected set of covariates such as gender, race, and other demographic variables. This approach provides a more rigorous estimation of the causal effects of incarceration on income and offers potential insights for policymakers on effective rehabilitation and re-entry strategies. By a targeted analysis of the causal impact of incarceration on income, this study is crucial for developing informed policy interventions aimed at mitigating the economic disadvantages faced by formerly incarcerated individuals.



\section{Data} %330
This research utilises the National Longitudinal Survey of Youth 1979 (NLSY79), a panel survey conducted by the U.S. Bureau of Labor Statistics. The NLSY79 is designed to collect data at multiple time points on the labor market activities and other significant life events of individuals born between 1957 and 1964, such as \textit{race}, \textit{gender}, \textit{age},etc. This dataset was specifically chosen for its longitudinal nature, which allows for collecting the incarceration data over time, and also because it includes detailed data on income. A critical aspect of the NLSY79 dataset for this research is the incarceration data. Incarceration dates are not directly asked in the survey. Instead, incarceration status is inferred from the \textit{Type of Residence} variables, which are part of the Household Composition section of the dataset. This section records if a respondent is in \textit{jail} or \textit{prison} at the time of the interview. This method of recording incarceration status presents a limitation, as it captures only those individuals incarcerated during the survey period and not those who may have been incarcerated between survey years. This could potentially underestimate the duration and frequency of incarceration experiences among the surveyed individuals. The NLSY79 employs a stratified random sampling technique, ensuring a representative sample of the U.S. population. From 2018 survey year dataset, key variables such as \textit{income}, \textit{prison} (for inferring incarceration status), \textit{race}, \textit{gender}, \textit{age}, \textit{urban}, and \textit{occupation} were extracted. \par

\subsection*{Variable Descriptions}
\begin{itemize}
    \item \textit{income} -- Total net family income in the previous calendar year.
    \item \textit{incarceration} -- Indicates whether the individual is in jail at the time of the survey.
    \item \textit{prison} -- Dummy variable equal to 1 if the individual has been sentenced at any time during the period from 1980 to 2018.
    \item \textit{race} -- There are three categories:
    \begin{itemize}
        \item \textit{hispanic} -- Dummy variable where 1 represents individuals who are Hispanic.
        \item \textit{black} -- Dummy variable where 1 represents individuals who are Black.
        \item \textit{non-hispanic and non-black} -- Represents individuals who are neither Hispanic nor Black.
    \end{itemize}
    \item \textit{gender} -- Dummy variable where 1 represents male individuals.
    \item \textit{urban} -- Dummy variable where 1 represents individuals living in an urban area.
    \item \textit{age} -- Age of the individual in the survey year.
    \item \textit{occupation} -- There are 13 types of jobs; each job type has its dummy variable (except for "unemployed" which may be treated as the base category). The occupations are:
    \begin{itemize}
        \item \textit{interviewer}
        \item \textit{teaching}
        \item \textit{sales and customer services}
        \item \textit{administrative assistant}
        \item \textit{medical administrator}
        \item \textit{nursing}
        \item \textit{insurance}
        \item \textit{small business owner}
        \item \textit{self employed}
        \item \textit{other professional}
        \item \textit{manager, supervisor, director}
        \item \textit{social worker, non-profit}
        \item \textit{government}
    \end{itemize}
\end{itemize}

 There are initially 12,686 observations. After cleaning to remove non-applicable responses such as ”Refusal”, ”Don’t Know”, ”Invalid Skip”, ”VALID SKIP” and ”NON-INTERVIEW”,  4670 observations remained for analysis. The table below shows the summary statistic of the cleaned dataset. \par

\begin{table}[htbp]
    \centering
    \caption{Summary Statistic}
    \label{tab: Summary Statistic}
    \begin{tabular}{lccccc}
        \toprule
        \textbf{Variable} & \textbf{Mean} & \textbf{Standard Deviation} & \textbf{1st Quartile} & \textbf{Median} & \textbf{3rd Quartile} \\
        \midrule
        income & 92665 & 110074.8 & 27007 & 63750 & 117905 \\
        prison & 0.05589 & 0.2297312 & 0 & 0 & 0 \\
        age & 57.23 & 2.225858 & 55.00 & 57.00 & 59.00 \\
        hispanic & 0.1927 & 0.3944769 & 0 & 0 & 0 \\
        black & 0.2961 & 0.4566049 & 0 & 0 & 1\\
        male & 0.4833 & 0.4997745 & 0 & 0 & 1 \\
        urban & 0.7942 & 0.4294881 & 1 & 1 & 1 \\
        unemployed & 0 & 0 & 0 & 0 & 0 \\
        interview & 0.3415 & 0.4742775 & 0 & 0 & 1 \\
        teaching & 0.06959 & 0.2544873 & 0 & 0 & 0 \\
        sales & 0.1422 & 0.3492763 & 0 & 0 & 0 \\
        administrative & 0.1281 & 0.3341828 & 0 & 0 & 0 \\
        medical administrator & 0.05054 & 0.2190703 & 0 & 0 & 0 \\
        nursing & 0.04026 & 0.1965823 & 0 & 0 & 0 \\
        insurance & 0.04754 & 0.2128083 & 0 & 0 & 0 \\
        small business owner & 0.03576 & 0.1857115 & 0 & 0 & 0 \\
        self employed & 0.005996 & 0.07720781 & 0 & 0 & 0 \\
        manager & 0.07088 & 0.2566483 & 0 & 0 & 0 \\
        social worker & 0.02313 & 0.1503208 & 0 & 0 & 0 \\
        government & 0 & 0 & 0 & 0 & 0 \\
        \bottomrule
    \end{tabular}
\end{table}

From the Table 1, we find that average income within the sample is \$92,665, with a notably high standard deviation of \$110,074.8. This significant variance and the disparity between the mean income and the median income suggest a skewed distribution, indicating the presence of high earners and a considerable degree of income inequality among the respondents. The proportion of interviewee who has been sentenced is about 5\%. The dataset includes substantial representation from racial minorities, with 19.27\% Hispanic and 29.61\% Black respondents. This diverse racial composition is beneficial for analysing the effects of incarceration across different racial groups. Interviewers represent the largest occupational group at 34.15\%, likely reflecting the nature of the dataset where many respondents are engaged in survey-related activities. Other notable occupations include sales and customer services (14.22\%), administrative assistants (12.81\%), and managers, supervisors, or directors (7.088\%). Notably, there is the absence of respondents categorised as unemployed or employed in government roles, which might indicate either a classification issue or a genuine characteristic of the dataset. These initial findings from the summary statistics highlight the need for further analysis, particularly to ensure that the propensity score matching accounts for the observed heterogeneity in income. \par

\section{Methodology} % relating data 114 + 200
The primary objective of this study is to determine the causal effect of incarceration on future income using propensity score matching (PSM). In observational data like the NLSY79, individuals are not randomly assigned to treatment (incarceration) and control groups, leading to potential selection bias. Incarceration is likely correlated with numerous observed and unobserved factors, such as socioeconomic status, that also impact income. PSM addresses this by matching individuals who have been incarcerated with those who have not based on a comprehensive set of observed covariates. This approach mimics randomisation by ensuring that the matched individuals are statistically similar in terms of these covariates, thereby isolating the effect of incarceration from these confounding factors. Additionally, the NLSY79 dataset includes a wealth of demographic and socioeconomic variables, such as age, race, gender, and urbanicity, which are critical in predicting both the likelihood of incarceration and income levels. PSM leverages this rich set of covariates to create a balanced comparison group, thus improving the reliability of the causal estimates by reducing the impact of overt biases. Therefore, this PSM method is chosen due to its strength in reducing selection bias and utilisation of rich covariate information in observational studies. 
\par
To make sure our matching could reveal the causal effect, we assume the conditional independence assumption (CIA), that is $Y_{ji}\perp\kern-5pt\perp D_i \mid X_i$. With CIA, the potential incomes are independent of incarceration status, conditional on the covariates $X$. Hence, we could eliminate the selection bias and estimated the causal effect. This assumption is likely to hold if we have sufficiently included all relevant covariates that influence both the risk of incarceration and income. Furthermore, we utilise common support assumption, which ensures that for every individual in the treatment group, there is a comparable individual in the control group, and vice versa. We validate this assumption by checking the overlap in the distribution of propensity scores across the treatment and control groups. The third one is Stable Unit Treatment Value Assumption (SUTVA), which assumes no interference between units, meaning the treatment status of one individual does not affect the outcomes of another. This assumption is generally reasonable in studies of incarceration and income, as the treatment (incarceration) is individually administered. 
\par

To implement the research, there are two steps. The first step is to calculate the propensity score of each individual, $p(X_i)$, which represents the probability of being incarcerated (treatment), that is $P[D_i=1|Y_{i}, p(X_i)]$. $X_i$ is the covariates vector, including age, race, gender, occupation and living area of individual $i$. These covariates are selected based on their potential influence on both the likelihood of incarceration and income levels. The propensity scores are derived through logistic regression, where the binary outcome is the incarceration status. $Y_i$ is the income. Following the calculation of propensity scores, individuals are matched from the treatment and control groups who have similar scores. This matching process helps to compare similar individuals, differing primarily in their incarceration status, to isolate the effect of incarceration on income. In this research, we estimate two treatment effects: Average Treatment Effect (ATE) and Average Treatment Effect on Treated (ATT). ATE is the average effect of incarceration on income across all individuals in the sample. ATT focuses on the average effect for those who have been incarcerated, comparing their observed outcomes to their potential outcomes had they not been incarcerated. 
\par
While PSM is a robust method for dealing with observable confounding, it does not account for unmeasured confounding factors. The NLSY79 dataset, while comprehensive, may lack certain critical variables such as detailed educational background or psychological factors that could influence both the likelihood of incarceration and subsequent income trajectories. The omission of such variables could introduce bias into our estimates, representing a significant limitation of this approach. Furthermore, PSM operates under the assumption that all relevant confounders are observed ('no hidden bias'), an assumption that is particularly strong and may not fully hold in our analysis. This limitation is crucial in the context of incarceration studies, where unobservable personal or environmental factors may play a significant role. However, within the scope of available data, PSM offers a feasible approach to approximate causal relationships, although the results should be interpreted with caution. 
\par







\section{Results} % score + att + balance check 350
\subsection{PSM Estimation}
The logistic regression was performed to estimate the propensity scores, which predict the likelihood of an individual being incarcerated based on selected covariates. These scores are crucial for the subsequent matching process used to estimate causal effects. The result of logistic regression is demonstrated in the table below.
\par 
\begin{table}[htbp]
    \centering
    \caption{Logistic Regression}
    \label{tab: Logistic Regression}
    \begin{tabular}{lcccc}
        \toprule
        \textbf{Variable} & \textbf{Estimate} & \textbf{Standard Error} & \textbf{z value} & \textbf{$Pr(>|z|)$} \\
        \midrule
        intercept & -0.90838 & 1.82181 & -0.499 & 0.61805 \\
        hispanic & 0.94583 & 0.20185 & 4.686 &  2.79e-06 *** \\
        black & 1.66467 & 0.16578 & 10.041 & \textless 2e-16 *** \\
        male & 2.56077 & 0.21858 & 11.715 & \textless 2e-16 *** \\
        urban & 0.47529 & 0.19100 & 2.488 &  0.01283 * \\
        age & -0.08585 & 0.03120 & -2.752 & 0.00593 ** \\
        interview & 0.05426 & 0.33144 & 0.164 & 0.86995 \\
        teaching & -0.68577 & 0.45557 & -1.505 & 0.13225 \\
        sales & -0.38646 & 0.36323 & -1.064 & 0.28735 \\
        administrative & -0.37754 & 0.36974 & -1.021 & 0.30722 \\
        medical administrator & -0.12231 & 0.41733 & -0.293 & 0.76947 \\
        nursing & -1.22639 & 0.60314 & -2.033 & 0.04202 * \\
        insurance & -0.39018 & 0.45050 & -0.866 & 0.38644 \\
        small business owner & 0.35164 & 0.43496 & 0.808 & 0.41883 \\
        self employed & -0.74606 & 1.09397 & -0.682 & 0.49525 \\
        manager & -0.02133 & 0.38867 & -0.055 & 0.95624 \\
        social worker & -0.87572 & 0.67490 & -1.298 & 0.19444 \\
        \bottomrule
    \end{tabular}
\end{table}

The logistic regression results indicate significant predictors of incarceration include race (Hispanic, Black), gender (male), and urban residency, with notable coefficients suggesting higher odds of incarceration for these groups. However, variables like occupation (teaching, sales, administrative) show less influence or non-significance in predicting incarceration status.
Given the large number of repeated propensity scores and that only about 5\% of individuals belong to the treatment group (incarcerated), we employed two different methods for handling ties in matching: average tie matching and randomly broken tie matching, enhancing the reliability of our causal estimates through repeated calculations. % randomly borken tie is not used in r

 \par
Using the above matching techniques, we observed a significant negative impact of incarceration on income, as summarised in the following table 3:
\par
\begin{table}[htbp]
    \centering
    \caption{Causal Effect Estimation}
    \label{tab: Causal Effect Estimation}
    \begin{tabular}{lcccc}
        \toprule
         \textbf{Causal Effect} & \textbf{Estimate} & \textbf{AI Standard Error} & \textbf{t value} & \textbf{p value} \\ 
        \midrule
         ATE & -59395 & 12580 & -4.7213 & 2.3432e-06  \\
         ATT & -59922 & 4610.7 & -12.996 & \textless 2.22e-16 \\
        \bottomrule
    \end{tabular}
\end{table}
Table 3 presents the estimation results. The ATE is estimated to be -59395, indicating that imprisonment is expected to reduce an individual’s yearly family income by \$59,395. The ATT is estimated at -59922. This suggests that the income of individuals who have been incarcerated is expected to be \$59,922 lower compared to the scenario where they were not sentenced. These extremely small p-values allow us to reject the null hypothesis with a high degree of confidence, affirming that the negative impact of incarceration on income is statistically significant. In addition, the use of Abadie-Imbens standard errors, which are designed to provide correct coverage in propensity score matching scenarios, adds robustness to our findings. This method ensures that our confidence intervals accurately reflect the true level of uncertainty in the estimates, affirming the reliability of our results. These findings underscore the profound economic consequences of incarceration, supporting the need for policies that address these economic disparities. \par

\subsection{Balance Check}
To ascertain the reliability of our matching procedure and ensure that it reveals an accurate causal effect, it was necessary to check for balance in the propensity scores across the treatment and control groups. Proper balance minimises selection bias and verifies that covariates are equally distributed between the groups. The table below demonstrates the result of balance check for ATE estimation. \par
\begin{table}[htbp]
    \centering
    \caption{Balance Check - ATE}
    \label{tab: Balance Check - ATE}
    \begin{tabular}{lcccc}
        \toprule
         & \textbf{Before Matching} & \textbf{After Matching}  \\ 
        \midrule
         mean treatment & 0.15446 & 0.055904  \\
         mean control & 0.050054 & 0.055889  \\
         std. mean diff & 114.01 & 0.020821 \\
         \\
         mean raw eQQ diff & 0.10393 & 0.00027495 \\
         med raw eQQ diff & 0.11562 & 8.4409e-06 \\
         max raw eQQ diff & 0.18141 & 0.027608 \\
         \\
         mean eCDF diff & 0.32745 & 0.0036915  \\
         med eCDF diff & 0.34597 & 0.0015038 \\
         max eCDF diff & 0.55738 & 0.036267 \\
         \\
         var ratio (Tr/Co) & 1.804 & 0.99844 \\
         T-test p-value & \textless 2.22e-16 & 0.3901 \\
         KS Bootstrap p-value & \textless 2.22e-16 & \textless 2.22e-16 \\
         KS Naive p-value & \textless 2.22e-16 & 6.9707e-07 \\
         KS Statistic & 0.55738 & 0.036267 \\        
        \bottomrule
    \end{tabular}
\end{table}
The propensity score is supposed to be balanced after matching in order to reveal the causal relationship. From the Table 4, we find that the mean of propensity score in treatment and control group are 0.055904 and 0.055904 respectively after matching, where difference decreases a lot compared with that before matching. Besides, the difference in empirical QQ and empirical CDF also becomes almost 0 after matching. This could be a support of covariate balance. The T-test p-value after matching is 0.3901, which implies there is no significant difference in means between matched groups at 5\% significance level, reinforcing covariate balance. However, the p-value for KS test is lower than 0.05, indicating that treatment and control group come from distinct distributions. In balance checking, the KS test result is preferred because it could provide the information about the distribution but not just the sample mean. Therefore, from the table 4, the covariates may not have a robust balance distribution, leading into invalid causal inference.
\par

Similarly, we could also prove that there is no robust covariate balance in ATT estimation. The balance checking result in illustrated in the Table 5 below. \par
\begin{table}[htbp]
    \centering
    \caption{Balance Check - ATT}
    \label{tab: Balance Check - ATT}
    \begin{tabular}{lcccc}
        \toprule
         & \textbf{Before Matching} & \textbf{After Matching}  \\ 
        \midrule
         mean treatment & 0.15446 & 0.15446  \\
         mean control & 0.050054 & 0.15446  \\
         std. mean diff & 114.01 & -0.0048908 \\
         \\
         mean raw eQQ diff & 0.10393 & 4.0359e-05 \\
         med raw eQQ diff & 0.11562 & 0 \\
         max raw eQQ diff & 0.18141 & 0.0014635  \\
         \\
         mean eCDF diff & 0.32745 & 0.0029578 \\
         med eCDF diff & 0.34597 & 0.00068997 \\
         max eCDF diff & 0.55738 & 0.046918 \\
         \\
         var ratio (Tr/Co) & 1.804 & 0.99996 \\
         T-test p-value & \textless 2.22e-16 & 0.64459 \\
         KS Bootstrap p-value & \textless 2.22e-16 & \textless 2.22e-16 \\
         KS Naive p-value & \textless 2.22e-16 & 0.0001394 \\
         KS Statistic &  0.55738 & 0.046918 \\        
        \bottomrule
    \end{tabular}
\end{table}
For ATT estimation, even though the T-test has a high p-value, low KS p-value still justifies the covariate imbalance. \par

\subsection{Genetic Matching}
Given the potential biases identified during the balance check for our initial ATE and ATT estimations, we implemented genetic matching as an advanced method to further refine the balance of propensity scores between treated and control groups. Genetic matching is a semiparametric statistical method that optimises the matching of treated and control units based on a genetic search algorithm, improving the quality of the matches. The genetic matching process was applied to reassess the ATE and ATT, and the results are displayed in Table 6. We find that ATE and ATT estimations are quite similar when using genetic matching, presented in the tables below. This approach provided improved balance across covariates, as evidenced by the updated balance checks and the resulting estimates of the treatment effects.\par
\begin{table}[htbp]
    \centering
    \caption{Genetic Matching Estimation and Balance Check}
    \label{tab: Genetic Matching Estimation}
    \begin{tabular}{lcccc}
        \toprule
         \textbf{Causal Effect} & \textbf{Estimate} & \textbf{AI Standard Error} & \textbf{t value} & \textbf{p value} \\ 
        \midrule
         ATE & -63389 & 4957.6 & -12.786 & \textless 2.22e-16  \\
         ATT & -63066 & 4916.2 & -12.828 & \textless 2.22e-16 \\
        \bottomrule
    \end{tabular}
\end{table}

\begin{adjustbox}{width=1\textwidth}
    \centering
    \label{tab: Balance Check - Genetic Matching}
    \begin{tabular}{lcccccc}
        \toprule
         & & \textbf{ATE} & & & \textbf{ATT} & \\
         & \textbf{Before Matching} & & \textbf{After Matching} &\textbf{Before Matching} & & \textbf{After Matching} \\ 
        \midrule
         mean treatment & 0.15446 & & 0.15446 & 0.15446 & &  0.15446  \\
         mean control & 0.050054 & & 0.15445 & 0.050054 & & 0.15445  \\
         std. mean diff & 114.01 & & 0.0050871 & 114.01 & & 0.0049434 \\
         \\
         mean raw eQQ diff & 0.10393 & & 6.607e-06 & 0.10393 & & 7.0165e-06  \\
         med raw eQQ diff & 0.11562 & & 0 & 0.11562 & & 0\\
         max raw eQQ diff & 0.18141 & & 0.0014635 & 0.18141 & & 0.0014635 \\
         \\
         mean eCDF diff & 0.32745 & & 0.00021002 & 0.32745 & & 0.00056756 \\
         med eCDF diff & 0.34597 & & 0 & 0.34597 & & 0 \\
         max eCDF diff & 0.55738 & & 0.0058161 & 0.55738 & &  0.013117 \\
         \\
         var ratio (Tr/Co) & 1.804 & & 0.99992 & 1.804 & & 0.99993 \\
         T-test p-value & \textless 2.22e-16 & & 0.60876 & \textless 2.22e-16 & & 0.61898 \\
         KS Bootstrap p-value & \textless 2.22e-16 & & 1 & \textless 2.22e-16 & & 0.924 \\
         KS Naive p-value & \textless 2.22e-16 & & 1 & \textless 2.22e-16 & & 0.96444 \\
         KS Statistic & 0.55738 & & 0.0058161 & 0.55738 & & 0.013117 \\        
        \toprule
    \end{tabular}
\end{adjustbox}
\par

We observe that, under genetic matching, both T-test and KS test p-value are far higher than 0.05. The standard mean differences, empirical quantile differences, and cumulative distribution function differences have all been minimised, indicating an optimal balance. The use of genetic matching has successfully mitigated the concerns raised by the initial balance check, providing robust support for the estimated causal effects of incarceration on income. These results confirm that individuals who have been incarcerated face significant economic disadvantages, with reductions in annual income exceeding \$63,000. This enhanced methodological approach underlines the strength of our findings and the utility of genetic matching in complex causal analyses. \par


\section{Conclusion} % 250
This study utilised propensity score matching to investigate the economic impact of incarceration, specifically how being sentenced influences income. Our analysis revealed that incarceration significantly reduces income, with the average treatment effect (ATE) estimated at -\$59,395 and the average treatment effect on the treated (ATT) at -\$59,922. Both estimates are statistically significant, affirming with high confidence that incarceration has a detrimental effect on income.
Initially, direct matching based on propensity scores indicated potential selection bias due to an imbalanced distribution of covariates between treated and control groups. To address this, we employed genetic matching, which refined the balance and produced robust estimates of ATE and ATT at -\$62,708 and -\$63,066, respectively. These adjusted estimates passed rigorous balance checks, underscoring the reliability of our conclusions.
Our findings align with the broader body of research that documents the negative socioeconomic consequences of incarceration. Studies such as \cite{western2000effects} and \cite{pager2003mark} have similarly reported long-term income penalties among formerly incarcerated individuals. Our study contributes to this literature by quantifying the income reduction and demonstrating the utility of advanced matching techniques to isolate the causal impact of incarceration. The significant income reduction associated with incarceration highlights the need for targeted policy interventions. Programs designed to facilitate the economic reintegration of formerly incarcerated individuals could include job training, education, and placement services that specifically address the barriers to employment this population faces. While our study provides valuable insights, it has limitations that future research could address. The reliance on self-reported data and the inability to capture all periods of incarceration might underestimate the true economic impact. Furthermore, our analysis could be enhanced by including more granular data on employment history and post-incarceration support services, which were not available in the current dataset. Additionally, further studies could explore the differential impacts of incarceration based on variables such as race, gender, and the nature of the offense, to better tailor policy interventions. Research could also assess the long-term impacts of incarceration on wealth accumulation and intergenerational economic mobility.
In conclusion, this study confirms the substantial economic disadvantage imposed by incarceration and underscores the urgency of developing policies that mitigate these effects while addressing the broader challenges of criminal justice reform.
 \par

\bibliographystyle{agsm}
\bibliography{name}

\end{document}
